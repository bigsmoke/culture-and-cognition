\documentclass{article}
\usepackage[utf8]{inputenc}
\usepackage[round]{natbib}
\usepackage{makeidx}
\usepackage{hyperref}

\title{Minor Culture and Cognition\\Notes and Summary}
\author{Rowan Rodrik van der Molen}

\begin{document}

\maketitle

\tableofcontents

\section{Evolution of the humanities}

The humanities as a field of study started in the 19th century, in response to
the rise of nation states and atheism, as a new style of group identity.  A
recurring theme that runs through the humanities is that culture and humanity
are somehow special, apart from nature, and thus not subject to study by the
natural sciences. To the critical observer, such a stance reeks a bit of
creationism.

The humanities are losing money; they're having trouble legitimizing themselves
as science proper while the naturalists are encroaching in their territory.
Therefore, there is increasing interest in multidisciplinarity. Hence the
research group that organizes this minor, which bridges the gap between
‘nature’ and ‘culture’---between brain matter and mind.


\citet{mesoudi2006}


\section{An interdisciplinary approach to the evolution of culture and cognition}

Biology, neuroscience, psychology, ethology, anthropology

Behaviourism (mind as a black box, studying inputs and outputs) was followed by cognitive science in 1957. Later came Noam Chomsky with his Universal Grammar / language device. His were the beginnings of theory, but not yet evolutionary theory.

Darwin's thesis: 4 requirements for evolution: variation, selection, heridity, continuity.

\citet{donald1991} offers a \textbf{gradualist} hypothesis, wherein the language faculty is not a discontinuos trait as it is in Noam Chomsky's \textbf{Universal Grammar} Hypothesis. The latter is a \textbf{saltationist} (\textit{salto}=big jump) approach to language. Saltationism, like \textbf{creationism}, is a \textbf{discontinuity} approach, and, like creationism, it violates one of the basic requirements for an evolutionary theory: \textbf{continuity}, which Donald's hypothesis does support.
\index{Universal Grammar}
\index{continuity}
\index{gradualism}
\index{saltationism}
\index{creationism}
% I'm not sure if my above statements are really accurate. Chomsky doesn't strike me as the discontinuous type.

Corballis: generativity $\ne$ discontinuity. Generativity: limited set of rules --> Unlimited set of phenomena.

\section{Levels of cognition}

Memory is a set of representations. “The human mind evolved from the primate (episodic) mind through a series of major adaptations, each of which lef to the emergence of a new representational system” \citep{donald1991}. The translation of memory to behaviour has become increasingly complex with each new level of representation.

\citet{donald1991} describes 4 stages of ape/hominid culture, each of which represents a new level of cultural complexity that emerges at the critical threshold of a gradual increase in cognitive abilities:

\begin{enumerate}
  \item episodic culture (non-semiotic), also present in the other extant great apes, emerged from the ability to represent and relate to events in memory;
  \item mimetic culture, allowed by iconic semiosis, emerged from the ability to model the self such that current behaviour can be linked to past, current or future events (metacognition);
  \item mythic culture emerged from the ability to invent semiotic devices (mind-tools/symbols); and
  \item theoretic culture emerged from our argumentative ability.
\end{enumerate}

Each new level of cultural complexity has depended on the evolution of new . “Each successive new representational system has remained intact within our current mental architecture, so that the modern mind is a mosaic structure of cognitive vestiges from earlier stages of human emergence.” \citep{donald1991}

\citet{collins2013} has a simpler take: System 1 ---> System 2

\index{semiosis}
Semiosis = the use of signs, the basis of communication.

\subsection{Episodic culture}

The ability to represent complex events in memory.

\subsection{Mimetic culture}

Memory needs ‘accurate’ representation that is continuously updated. In episodic
cognition, these memories are directly transformed into behaviour. Mimetic
cognition adds a self-referential layer so that the representation can be
turned in behaviour and visa versa.  \textbf{Mimesis} is the ability to act out
episodes in your memory to others.  This is made possible by
\textbf{metacognition}, or \emph{supramodel capacity}---a representation in
memory of how events are represented in memory.  Metacognition adds a
self-referential layer on top of episodic cognition, so that, besides acting
\emph{out of} representations (as in the episodic mode), these representations
can be \emph{acted out} to others or to ourselves for further self-reflection
(ad recursum). We can recognize our own memory and merge representation and
behaviour into one event. Also we can do this collectively, which allows human
groups to represent themselves.

Due to their mimetic abilities, our forebears entered into mimetic culture,
which was (and is) characterized by distributed cognition. Emulating others
and re-enacting (social) situations tightened social organization and allowed
increasing social complexity so that a \textbf{hybrid mind}---a mix of
biological and cultural programming---was achieved. Thus, against Swaab's thesis,
we are not (just) our brain. 

Enables tool-use \& manipulation.

Aristotle on tragedies: Imitation is good
Katharsis requires a certain distance between the work of art/artist and the viewer as well as identification. Memesis allows us to empathize. [According to Miranda, that is. I know for a fact that chimps can empathize, and their culture is episodic.]

An example of mimesis: using a water bottle to signify a spear.

Mimetic behavior: motor patterns (ritual/rhythm).

Mimesis allow the use of mind tools to interact with each other instead of just with the world. [Does it?]

Art is a type of cognitive engineering tool to manipulate the minds of others.

A poet plays with the space between literal re-enactment and convention by turning words into objects---the process of femaliarization/enstrangement (Heidigger).

\subsection{Group selection}

Donald argues that, with the advent of mimetic culture, selection no longer operated solely at the level of the individual, because group selection came into play.

In distributed cognition, the individual is the layer that connects the group to the world.

\subsection{Mythic culture}

Apes can be trained in symbol use, but they don't invent symbols. This requires more than a capacity for linking signs and signifiers \citep[p.~217-220]{donald1991}. A symbol can be considered a mind-tool \citep{gregory1981} and ``must have followed an advance in thought skills'' \citep[p.219]{donald1991}.

Mimetics: perceptual mental models

Symbols: conceptual mental models

Causes of mimetic ---> mythic transition:
\begin{itemize}
  \item Speech adaptation
  \item Not geographic or climatic conditions
  \item Competition between groups of hominids instead of individuals % ESSAY
  \item Primary human adaptation was mythical thinking, not language. Symbols were needed for these new mental models.
\end{itemize}

Group formation as the selection pressure on myth development.

Narrative---series of events (in contrast to re-enacting 1 event as in mimetic culture)

Mythic behaviour: representations ---> Language (myth/stories/concepts) A concept is a label.
“The ability to abstract”?
More efficient representation.
More generalized communication.

Myth originally relied entirely on biological memories. Shamans where in posession of the greatest biol. memory, an early---perhaps \emph{the} earliest---form of specialization.

Primary functions of narrative in mythic culture: 
\begin{itemize}
  \item Origins of group as a whole
  \item Sharing of knowledge
  \item Gossip was (and is) a very important method for figuring out social relationships that is much more effecient than the one-to-one method of social exploration that is the only avenue of collecting social knowledge in non-human primates.
  \item Sharing of wisdom (ethical values and meanings)
\end{itemize}
Mythic invention: “Construction of conceptual models of the human universe”

Religion: “Belief in and worship of a superhuman controlling power, especially a personal God or gods” (Oxford dict., modern def.)

Religion is also the use of narratives, rituals and symbols. Did mimetic culture already feature a type of protoreligion?

Religion comrises myths, but not exclusively; it also comprises rituals (mimetic/emotional experiences).
Relgion combines mythic stories with mimetic rituals.

You could say that rituals are on the interface of mimetic and mythic culture. Mythic culture extends rituals with symbolism and narrative.

Religions are empowering/dispowering penomena.

\textbf{Rituals} revolve around the acting out of events. Rituals are collective acts ‘designed’ to provide the participants with a shared cognitive model of society. 

\textbf{Ideological State Apparatuses} (\textbf{ISAs}) are organizations such as government and religions (Louis Althusser, 1918-1990)

Use of Heidegger by Collins is wrong (Ch. 3 in paleopoetics).

\subsubsection{Structuralism}

Saussure (1857-1913).

signifi\'{e} = phonetic aspect of the sign
significant = semantic concept
Binary oppositions: chair $\ne$ table

\subsubsection{Linguistic turn}

Mid 20th century.
Language as focal point  in understanding historical representation.
Subjectivity
Gender. Women were no longer considered merely as a peculiar type of men.

\textbf{Tropes} connect concepts in different ways and have a function in group formation, by seperating the hearers of message 1 and message 2.

\begin{enumerate}
  \item \textbf{Metaphor}
  \item \textbf{Metonymy}
  \item \textbf{Synecdote}
  \item \textbf{Irony}, the basis of humor, consists of a message that is clearly acted out and a (subliminal) message that is represented but not acted out. The subliminal message could be hidden in the tone of voice, posture, or in the mimetic representation of the event to contrast with the symbolic content of the verbal report.
\end{enumerate}

Meaning is created by \textbf{discourse}---“practices that systematically form the objects of which they speak (Foucault, 1926-1984).” Implications:
\begin{itemize}
  \item Anti-humanism (I only exist as a crossroads)
  \item Knowledge/power relation
  \item Linguistic being trapped in symbol space
\end{itemize}

\textbf{Deconstruction} (Derrida, 1930-2004): Unpack hierarchy of meaning implicit within dichotomous thinking.

\subsection{Theoretic culture}

Whereas mythic culture was fully dependent on the combined power of biological
memories, theoretic culture emerged due to the increasing use of artifacts
that could serve as an external representation of memory.

Theoretic behaviour: theory/science/philosophy

The ability to doubt/question \& speculate.

From maintenance of knowledge (in mythic \& mimetic culture) to maintenance of ``knowledge skills'' (in theoretic culture).

Natural intelligence, in theoretic culture, is mapped onto social intelligence through actual tools rather than through behaviour. [Don't like this formulation.]
(\textbf{Cognitive Fluidity}, Stephen Mithen)

\textbf{ExMF} = \textbf{External Memory Field}.
\textbf{ESS} = \textbf{External Symbolic Storage}.

2 modes of modern thinking: narrative thought (inherited from mythic culture) \& analytical thought.
Analytical thought exploded with the ancient Greeks. They were wealthy, free and they had the right tools: a phonetic script.

Theoretic attitude.

\textbf{Demythologization}: mythic ---> theoretic; myth ---> theory. Explanation: unpredictable ---> predictable.

Discovery of new cultures led to cultural relativity (and doubt). Also, astronomy and geometry were imported.

Speculative philosophy: thinking for its own sake, not for semiotic purposes.

Rational/reflective criticism (argumentative) => Quick succession in theories.

Greek philosophy branched out into natural science and cosmology, mathematical proof, \textbf{aetiological} theories (`origins' myths reworked as theories)

The `highest' form of analytical thought is formal theory.

\section{Speech}

Prosody

Phonology
Parralel with prosody system \citep[p.~249]{donald1991}.

In pure Wernicke's aphasia, phonology and morphology of language remains intact, whereas semantics and syntax suffer. \citep[p.~249]{donald1991}.

lexicons (vs dictionaries) \citep[fig.~7.1, p.~251]{donald1991}.

Two major modes of thought: narrative and paradigmatic.

The myth as the supreme product of the narrative mode.

\section{Brain structures}

\paragraph{Aphasics}

\paragraph{Wernicke's region}

Different effects of damage to this region \citep[p.~262]{donald1991}: agrammatism and reading+writing.

Phonology is highly laterized to the left hemisphere \citep[p.~262]{donald1991}.

\section{Biological (brain) evolution}
\label{sec:biological_evolution}

\textbf{The default hypothesis} is the hypothesis that typical human (system 2) features
of the brain evolved in the left side of the brain (Korbalis).
unlike
\begin{tabular}{rl}
5 mya & common ancestor homonids \& chimps \\
4 mya & \textit{Australopithecus} (“Southern ape”) \\
2 mya & \textit{Homo habilis} (“Handy man”) \\
1.5 mya & \textit{Homo erectus} \\
0.3 mya & (Archaic) \textit{Homo sapiens} \\
0.04 mya & \textit{H. sapiens sapiens} \\
\end{tabular}

\paragraph{Hominidae}

\paragraph{\textit{Homo habilis}} lived 2.3 to 1.6 mya and created the first manufactored tools (`Oldowan' tools).

\paragraph{\textit{Homo erectus}} lived 1.9 mya to 45 kya.
Oldowan and Acheuléen tools.

The development of an advanced vocal apparatus would have followed from the selection pressure from a mind that needed vocal language for its modeling \citep[p.220]{donald1991}.

\subsection{Four-dimensional evolution}

\citet{jablonka2007}


\section{Cultural evolution}
\label{sec:cultural_evolution}

\subsection{Tool use}
\label{sec:tool-use}



\bibliographystyle{plainnat}
\bibliography{evo}


%\makeindex

\end{document}
