\documentclass{article}
\usepackage[utf8]{inputenc}
\usepackage[round]{natbib}

\title{Minor Culture and Cognition\\Notes and Summary}
\author{Rowan Rodrik van der Molen}

\begin{document}

\maketitle

\tableofcontents

\section{The humanities}

\citet{mesoudi2006}

\section{Levels of cognition}

\citet{donald1991} describes 4 stages of ape/hominid culture, each of which represents a new level of cultural complexity that emerges from a gradual increase in cognitive abilities:

\begin{enumerate}
\item episodic culture, also present in the other extant great apes, emerged from the ability to represent and relate to events in memory;
\item mimetic culture emerged from the ability to model the self such that current behaviour can be linked to past, current or future events (metacognition);
\item mythic culture emerged from the ability to invent semiotic devices (mind-tools/symbols); and
\item theoretic culture emerged from our argumentative ability.
\end{enumerate}

\subsection{Episodic culture}


\subsection{Mimetic culture}

\subsection{Mythic culture}

Apes can be trained in symbol use, but they don't invent symbols. This requires more than a capacity for linking signs and signifiers \citep[p.~217-220]{donald1991}. A symbol can be considered a mind-tool \citep{gregory1981} and ``must have followed an advance in thought skills'' \citep[p.219]{donald1991}.

\subsection{Theoretic culture}

\section{Speech}

Prosody

Phonology
Parralel with prosody system \citep[p.~249]{donald1991}.

In pure Wernicke's aphasia, phonology and morphology of language remains intact, whereas semantics and syntax suffer. \citep[p.~249]{donald1991}.

lexicons (vs dictionaries) \citep[fig.~7.1, p.~251]{donald1991}.

Two major modes of thought: narrative and paradigmatic.

The myth as the supreme product of the narrative mode.

\section{Brain structures}

\paragraph{Aphasics}

\paragraph{Wernicke's region}

Different effects of damage to this region \citep[p.~262]{donald1991}: agrammatism and reading+writing.

Phonology is highly laterized to the left hemisphere \citep[p.~262]{donald1991}.

\section{Biological (brain) evolution}
\label{sec:biological_evolution}

\paragraph{Hominidae}

\paragraph{\textit{Homo habilis}} lived 2.3 to 1.6 mya and created the first manufactored tools (`Oldowan' tools).

\paragraph{\textit{Homo erectus}} lived 1.9 mya to 45 kya.
Oldowan and Acheuléen tools.

The development of an advanced vocal apparatus would have followed from the selection pressure from a mind that needed vocal language for its modeling \citep[p.220]{donald1991}.

\subsection{Four-dimensional evolution}

\citet{jablonka2007}


\section{Cultural evolution}
\label{sec:cultural_evolution}

\subsection{Tool use}
\label{sec:tool-use}



\bibliographystyle{plainnat}
\bibliography{evo}

\end{document}
