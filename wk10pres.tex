\documentclass{beamer}

% Week 10 Presentation
%
% Duration: 1 0min presentation + 5 min discussion
%
% Raising a sensible question is sufficient

\begin{document}

\title{The use and abuse of altruism}
\subtitle{(in thought and in the wild)}

\begin{frame}
  \frametitle{Contents of this talk}
  \tableofcontents
  
  You're probably not as noble as you think,
  except when it comes to conformity.
\end{frame}

\section{What is altruism?}

\begin{frame}
  \frametitle{A definition of altruism}
\end{frame}

\begin{frame}
  \frametitle{What altruism is not}
  
  
  \begin{itemize}
    \item Altruism is not not being a spiteful jerk.
    
    \begin{itemize}
      \pause
      \item \textbf{Spite} (scientific term): harm caused to another that doesn't increase the fitness of the perpetrator.
      \item Being such an asshole is not evolutionary stable.
    \end{itemize}
    
    \pause
    
    \item Altruism is not the same as \textbf{mutualism}.
    
    \begin{itemize}
      \item Both parties derive a benefit from mutualism.
    \end{itemize}
  \end{itemize}
\end{frame}

\section{Altruism as an ESS}

\begin{frame}
  \frametitle{The ESS}
  
  \textbf{ESS = Evolutionary Stable Strategy}
  (not te be confused with Donald's External Symbol Storage)
  
  Doves versus Hawks
\end{frame}

\begin{frame}
  \frametitle{Evolutionary explanations of altruism}

  
\end{frame}

\begin{frame}
  \frametitle{Kin selection}
\end{frame}

\begin{frame}
  \frametitle{Reciprocal altruism (tit-for-tat)}
\end{frame}

\begin{frame}
  \frametitle{Altruism as a side-effect}
\end{frame}

\section{Altruism and group-sacrifice}

\begin{frame}
  \frametitle{Group selection}
  
  \begin{itemize}
    \item Positive feedback loop: cultural evolution vs. ability to conform
    \item Group stability: adapters vs. innovators
    \item Group stability would undermine cultural stability
  \end{itemize}
\end{frame}

\begin{frame}
  \frametitle{Groups >> Peers}
  
  As a consequence of mimetic culture,
  
  \begin{itemize}
    \item groups have much outgrown the sum of a set of peers.
    \item Fitting in `your' group is of utmost essence for individual survival,
    \item much more so than getting a good ROI (return on investment).
    \item Demands of the group outweigh individual demands.
  \end{itemize}
\end{frame}

\begin{frame}
  \frametitle{Intergroup warfare}
  
  \begin{itemize}
    \item It's difficult to say if any group attribute can be said to be part of an ESS.
    \item It depends in part on whether \textit{Homo sapiens} is evolutionary stable.
    
    \item The metapopulation's culture will influence the group culture.
    \item The losing culture can assimilate the winning culture
    \item The winning culture can 
  \end{itemize}
\end{frame}

\begin{frame}
  \frametitle{Where does this leave us?}

  Who needs dark forces to enslave them,
  when all you need is a cold, heartless culture?
  
  Does it really matter if `our' culture wins wars, unless it is really \textit{ours}?
  It probably does, because it's no fun to be part of a group that is annihalated.

\end{document}
