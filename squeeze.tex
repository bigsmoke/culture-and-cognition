\documentclass{article}

\usepackage[utf8]{inputenc}
\usepackage[round]{natbib}

\title{Caught in the middle\\---\\between genes and memes}
\author{Rowan Rodrik van der Molen}

% The essay has to cover 3-4 book chapters/papers,
% besides the required reading for the course.
%
% Idea: use endnotes/footnote for concepts that Eelco may be less familiar with.
%
% The 1st and 2nd draft are also handed in.
%
% \section{Introduction}
% Introduce topic
%
% \section{Discussion}
% The main body of the text
% Main question: Should we include group selection in human pop. genetics?
% Use (only) 1-2 examples where individual-level selection doesn't apply.
%
% \section{Conclusion}
% should relate the discussion to the introduction

\begin{document}

\maketitle

\tableofcontents

\section{Introduction}
\label{sec:intro}

This essay attempts to interrupt the dominant cultural narrative that the human condition has been steadily improving ever since human cultures started increasing in complexity---an accelerating process that was kicked in a higher gear with every new social revolution (e.g.: the agricultural revolution and the industrial revolution). While trying to explain why our culture---that very thing which we derive meaning and identity from---doesn't care about us, I'll demonstrate why it would be in our genes' best interest to make us care more about our disinterested cultures than about our own well-being.

As soon will become apparent, we cannot escape the cages of our biological and cultural programming. But, we can recognize that, unlike our biological heritage, our cultural programming is not an immutable fact of life. Potentially, our culture can be reprogrammed to the advantage of shared (biological) values.

\section{Mimetic culture and group selection}
\label{sec:mimesis}

$$p^2 + q^2 + 2pq = 1, P=p^2, Q=q^2, R=2pq$$

\begin{eqnarray}
  \label{eqn:fitness}

  w_{abs,P}, w_
\end{eqnarray}

Integrate group selection as drift at the level of individual selection.

Group selection is about more than the cultural alleles. Tool-use and technology greatly complicate things.

\subsection{The selfish gene}
\label{sec:genes}

Selfish in respect to the individual, who may be forced into altruistic behaviour by such a gene.

\subsection{The selfish meme}
\label{sec:memes}

A meme, according to \citet{dawkins1976}, can be defined as any type of cultural entity that an observer might consider a replicator.

Memes are information that can be exchanged between brains and thus be transferred to the next generation.

\subsection{Mimesis and a new level of selection}

Mimetic culture \citet{donald1991}.

%Explain how we resemble ants in our interdependence, but that our behaviour is not genetically coded for.


%This is amplified by mythic culture

In population genetics, various levels of selection have been considered: gene-level, individual-level and group-level selection. Only individual-level selection is normally considered, although genes can always be considered selfish to some extent \citep{dawkins1976}. Mathematically, this only matters when genes are so selfish that they compete within an individual. Otherwise, it's more useful to think in terms of fitness not of a single gene, but in the relative fitness of an individual carrying that gene.

Group selection is hardly ever applied in practice, although it is recognized that there are some theoretical cases where it would apply.
% Theoretical cases where group selection applies

% natural selection,
% sexual selection,
% group selection,
% cultural selection

\section{The altruistic individual}

\subsection{Killing yourself may be good for your genes}

Suicide as an adaptive strategy.
% http://blogs.scientificamerican.com/bering-in-mind/2010/10/11/is-killing-yourself-adaptive-that-depends-an-evolutionary-theory-about-suicide/

\subsection{Murdering your neighbours may be good for your memes}

It is not controversial that some of your genes (i.e. those shared not by \emph{them}) may be well-served by murdering the competitors of your group.

\subsection{Positive feedback loops between cultural and biological evolution}

\subsection{The paper tiger and other cultural stresses}

How culture has long driven (groups of) people to excell, but has squasched curiosity.

\subsection{The end of altruism: successful psychopaths on the rise}

\section{The artful individual}

In this age of genetic engineering, can we re-engineer our culture?

Fourth-generation warfare.

\bibliographystyle{plainnat}
\bibliography{evo}


\end{document}
