\documentclass{article}

\usepackage[utf8]{inputenc}
\usepackage[round]{natbib}
\usepackage{amsmath}
\usepackage{hyperref}

\title{Caught in the middle\\---\\between genes and memes}
\author{Rowan Rodrik van der Molen\footnote{r.r.van.der.molen@student.rug.nl}}

% The essay has to cover 3-4 book chapters/papers,
% besides the required reading for the course.
%
% The 1st and 2nd draft are also handed in.

\begin{document}

\maketitle

\tableofcontents

\section{Introduction}
\label{sec:intro}

% The question of universal validity is really awkward

This essay explores the implications of being a cultural creature, the central
question being what it means to be a carrier of genes that have evolved to support
the existence and evolution of what \citet{dawkins1976} calls memes. This is
definitely a very personal question, for both you---the reader---and me---the
writer. Given that neither our genes nor our memes will be identical, there's 
unlikely to be an answer that is universally valid. What I can, however, do is
to frame the question such that it helps us to successfully navigate the
tensions that exist between our individual and group needs, our cultural
demands and the genetic programming of ourselves and other individuals.

The ultimate purpose of this exercise is to gain a deeper understanding of why
we do what we do and also (hopefully) to make it easier to recognize what
we want to do. As unlikely as the latter outcome may be, it is the desire to
know this---or rather, my frustration with the metacognitive deceptions that
have so far gotten in the way of true self-knowledge---that led to my long-term
fascination with this subject. It all started with a dream.

I had a dream---a lucid dream, in which I was aware that I was dreaming. From
studying \citet{laberge1990}, I knew that, in this dream-world, I could do
anything I wanted to. Nothing happened, until something did happen---something
that I thought I wanted; it turned out I didn't. The ensuing confusion---that I
could no longer trust my personal narrative about myself and my
motivations---arose from a misappropriation of the ownership of my personal
history, which was never personal to begin with, but had really always been a
cultural product. So how can culture come to possess a mind to the extent that
it no longer feels as our own?

\section{Discussion}

\paragraph{Mimetic culture}

\citet{donald1991} provides an account of the gradual evolution of modern
cognition that achieves its gradualism by hypothesizing an intermediate stage
between episodic cognition, which we share with the other great apes, and
mythic cognition, which, of the extant primates, only \textit{Homo sapiens}
possesses. He calls this intermediate stage \textit{mimetic} cognition.

In his view, mimesis---the ability to act out episodic
memories\footnote{Mimesis is not the same as mimicry, which involves the exact
  duplication of behaviours, such as the duplication of sounds in song birds.
  Nor is the same as imitation, by which apes and monkeys copy their parents'
  way of doing things.  “Mimesis usually incorporates both mimicry and
  imitation to a higher end, that of re-enacting and re-presenting an event or
  relationship. [\ldots] It involves the \textit{invention} of intentional
  representations” \citep{donald1991}.}---is still the basis of much of
contemporary culture. Rituals, games, customs and many skills are at least
partially mimetic in nature. Mimesis depends on advanced meta-cognitive
abilities that to some extent may be present in other mammals (some examples
of which can be found in the commentaries on \citealp[pp.~755--756, 768,
771-772]{donald1993}), but which Donald assigns to late \textit{Homo habilis}
or early \textit{Homo erectus} along with the first archaeological evidence of
tool making \citep{donald1991,donald1993}.

\citet[pp.~254, 338]{donald2001} suggests that from the mimetic stage onwards,
evolutionary models may be improved by including group selection as an
additional level of natural selection, running in parallel and interacting with
individual-level natural selection.  When the relative fitness of individuals
became intertwined with the culturally-programmed phenotype of their group,
groups turned into mediators between group-members and selective pressures
emanating from the group's environment, or \emph{niche}. The group's unique
culture came to influence the fitness of its members relative to members of
other groups, and, as the relative importance of selective forces emanating
from the group started to outweigh those emanating from the group's
environmental niche, it became increasingly important for genes to support the
behavioral flexibility required by rapidly evolving group cultures. Thus, the
possibilities of self-sacrificing and altruistic behaviour had to be extended
to include the group as an object of altruism.\footnote{With this broadening
  (read: easing) of the individual's moral community, problems regarding
  cheating will inevitably pop up, but, with group selection in full swing,
solutions to such problems can and will be solved by cultural evolution, while
social pressures on the genetics of group members will ensure compliance to
such solutions.}

Archaic hominids were already social (and altruistic) animals before the
emergence of mimetic culture, but only with the advent of mimetic culture did
the interests of the group start to override the interests of the individual.
How this was not so before deserves elaboration.

\paragraph{Selfish genes}

%\footnote{For a more thorough evolutionary analysis of altruism (which
%  nevertheless ignores the group selection debate), I would definitely
%  recommend \citet[ch.~12]{freeman2014}.}

For non-human animals, the argument that they've evolved to serve “the greater
good” is easily refuted, as has repeatedly been done by a number of authors
(e.g. \citealp{hamilton1964, dawkins1976}). \citet{dawkins1976} has done this
most expertly by means of reducing individuals to copying machines for their
genes. Genes, then, can encourage altruism whenever this serves them best. That
is, those genes that have evolved to---in concert---influence the behaviour of
their host such that the maximum number of these genes are present in the next
generation will be the genes that will preferentially survive. Altruism then
makes perfect sense if you consider that family members will typically share
many genes, and indeed altruistic tendencies are generally highly
discriminative, with the closest relatives being the most likely beneficiaries,
because the inclusive fitness value of altruistic behaviour is dependent on the
\textit{coefficient of relatedness} \citep{hamilton1964}. This type of altruism
is said to be the product of \emph{kin-selection} and can be seen in, for
example, the banded mongoose \citep{gilchrist2004a, gilchrist2004b, hodge2005}.

Another type of altruism that is fairly common in the animal kingdom is
reciprocal, or \emph{tit for tat}, altruism, proposed by \citet{trivers1971}
as a model for cooperation between unrelated individuals within groups
of primates, including humans. A requirement for this strategy, according to
Trivers, is that individuals possess a good enough memory for keeping track of who
owes them what and what they owe whom. Such detailed book-keeping is necessary
to avoid cheating; otherwise, cheaters \textit{will} inevitably pop up.
Thanks to their excellent episodic memory, nonhuman primates are definitely
capable of this type of altruism \citep{waal1996}.

In our own complex primate society, the functioning of psychopaths offers
insight into the social consequences of cheating. When so-called
\emph{successful psychopaths} are not detected, they can victimize not only
individuals, but whole groups, even economies \citep{babiak1995, boddy2006,
boddy2010, kirkman2005}. Obviously, for altruists, it's essential to be able to
weed out and discriminate against cheaters, or their altruism would not be an
evolutionary stable strategy (ESS) and remain vulnerable to invasion by
mutants. \citet*{axelrod1981} compared many possible strategies using game
theory and concluded that \emph{tit for tat} was always the most successful
strategy. It was also the simplest, based on three rules: a player following
this strategy (1) is never the first to defect; (2) provoked to immediate
retaliation by defection; and (3) willing to cooperate again after just one act
of retaliation for defection.

When \citet{dawkins1976} called our genes ‘selfish’, he meant that for them it
doesn't matter through which carrier their copy number is maintained. Genes
that act in concert to maximize their number in successive generations can
achieve a kind of immortality, regardless of which of their disposable vehicles
passes down these genes along generation lines.\footnote{Do note that, despite the
anthropomorphizing language use, genes do not intentionally partake in this
process.} The above-mention explanations for altruistic behaviour are fully in
accord with Dawkin's gene-centric view of evolution wherein the replicators are
genes and the vehicles are individual organisms. However, it can be argued that
humans have evolved an expanded facility for altruism that cannot be
satisfactorily explained by natural selection operating on individuals alone.

\paragraph{Group selection}

Not all authors agree that group selection is an extraneous explanation for
altruistic behaviour in nonhuman animals. So far, I've followed here the
consensus view as propagated by \citet{hamilton1964} and \citet{dawkins1976}.
Donald's suggestion---that mimesis might, in fact, imply group selection---is
buried in an endnote \citep[p.~338]{donald2001} probably because it refers to a
controversial work of \citet{sober1999} in which they try to resurrect group
selection as one level in their multilevel selection theory. (See
\citealp{dennett2002} for a decent critique of this model.)

For the moment, we can bypass most of the controversy by selectively applying the
concept of group selection to mimetic culture as per Donald's suggestion.
Ignoring the rest of the debate is perfectly acceptable, because for most
animals models with or without group selection are mathematically and
experimentally equivalent \citep{dennett2002}. Either way, within Donald's
framework, of which gene-culture co-evolution is a central feature, a model
that includes group selection makes sense.

% Also proposed by evolution in four dimensions
Multi-level selection models (such as proposed by \citealp{sober1999}) have
been criticized for failing to distinguish between replicators and vehicles
\citep{dennett2002}.  When we consider that human groups can (and often are)
extinguished or outcompeted by other groups, it becomes apparent that groups are
perfectly valid vehicles for the replicable units of cultural information called
\emph{memes} by \citet{dawkins1976}. The increased intergroup competition due
to an increase in (fitness) variation between mimetic group cultures could have
sped up the increase in brain size of \textit{H. erectus} on the way to
\textit{H. sapiens} and \textit{H. neanderthalensis} by means of a positive
feedback loop wherein increasingly demanding group cultures selected for
increasingly adaptive and clever individuals, who in turn could invent cleverer
cultural practices.

It is this accelerating phase of hominid evolution where we have to expect the
individual to have evolved to define its identity more and more in terms of
group position, an ongoing process which switched into a higher gear with the
advent of mythic culture.

\paragraph{Mythic culture}

Mythic culture emerged together with spoken language, probably in \textit{H.
sapiens}. Gossip and myth combined with ritual, allowing more complex forms of
social organization and communication. Group life became embedded in a
narrative, which supplied group members with meaning and identity \citep{donald1991}. 

Significantly, until the agricultural revolution, human groups remained
organized as small tribes. It is safe to assume that most modern humans, in the
8000--13000 years since the agricultural revolution took place in the fertile
crescent, haven't had the opportunity to fully adjust their social instincts to
the demands of life in contemporary, impersonal super-groups \citep{quinn1992}.

\paragraph{Selfish memes}

A meme, according to \citet{dawkins1976}, can be defined as any type of
cultural entity that an observer might consider a replicator.  Memes are
information that can be exchanged between brains and thus be transferred to the
next generation. An important feature of memes is that, as long as they can
‘infect’ enough new individuals through their host (horizontally or
vertically), they do not have to supply their host with a fitness benefit.

When combining a meme-centric view of cultural evolution with a
group-selectionist account of the supporting genetic changes, a grim picture
emerges: human culture, liberated from the confines of tribal groups, has
evolved to the point where it matters more how humans can be productively
harvested for the outputs advantageous to their culture than how they wish to
live together. And thus, our unique propensity for self-sacrifice enters the
picture, one which is not always entirely voluntary, but all too often deemed
necessary, because the alternative to sacrificing oneself for our misanthropic
memes is often to actually or effectively kill oneself.

Suicide (behavioral and physiological) is not uniquely human. What \emph{is}
uniquely human is that the unfolding of the narratives we live by can be
sufficient to convince us that our lives are no longer worthwhile
\citep{baumeister1990}. It is true that in some circumstances suicide may be
good for your genes, and kin selection can perhaps account for those instances
\citep{bering2010}. When, however, suicidal tendencies are solely caused by
mythical displacement (and not by actual obsolescence), it must be evident that
our genes are losing from our memes.\footnote{See \citet{quinn1992} for a unique
insight into the devastating psychological and ecological consequences of being
possessed by memes that do not acknowledge our genetically programmed needs and
limitations.}

\section{Conclusion}

\paragraph{Cliff-hanger}

Why is it so important to be able to distinguish between the various types
of self-sacrifice that have been discussed in this essay? Because, in this century,
our survival as a species might come to depend on our ability to reclaim culture
as the type of communal property as which it began instead of continuing to live
our lives as sacrificial lambs in the service of unfriendly memeplexes. Not
just us, but all the planet's ecological support systems have been converted by
culture. This \textit{enculturalization} process is still running on full
steam, blind and relentless as ever. There are two problems with this: (1)
anthropocenic\footnote{The \emph{anthropocene} is an informal geological term
for the current epoch that reflects the fact that much of our environment is
either completely man-made or transformed by human activity
\citep{revkin2011}.} cultural practises have pushed us fully into the sixth
great extinction event, and (2) this culture drives people crazy.

The first problem is widely recognized and uncontroversial \citep{iucn2009,
pimm1995}. The extent of the second problem is best summarized by Robert
Sapolsky \citeyearpar[ch.~14]{sapolsky2004}: “5 to 20 percent of us will suffer
a major, incapacitating depression at some point in our lives, causing us to be
hospitalized or medicated or nonfunctional for a significant length of time.
[The] incidence [of depression] has been steadily increasing for decades---by the year 2020,
depression is projected to be the second leading cause of medical disability on
earth.”

\paragraph{Harmful narratives}

This essay has attempted to interrupt the dominant cultural narrative that the
human condition has been steadily improving ever since human cultures started
increasing in complexity---an accelerating process that was kicked in a higher
gear with every new social revolution, such as the agricultural revolution and
the industrial revolution \citep{botton2013, quinn1992}. To explain why our
culture---that very thing which we derive meaning and identity from---doesn't
necessarily care much about us (and even less about our feelings), I've
demonstrated why it could be in our genes' best interest to make us care more
about our disinterested culture than about our own (and each other's) personal
well-being.

Regarding my own confusion about my identity---the confusion that prompted this
exercise---I can conclude that part of this confusion is due to a misappropriation
of what constitutes ‘my’ mind. To clarify: my mind is both the product and a
constituent of my \emph{community of mind}. As a product, I will unwittingly
try my best to fit in and conform to whatever demands are implied by whatever
myths I feel part of. But, as a constituent, unsatisfied as I am with my
programming, I will always try to sabotage and cheat my way out of a cultural
identity that will forever remain at odds with my genetically determined
sense of community. And most of this self-sabotage will remain unconscious as
long as my conscious self is occupied with trying to fit my thoughts into
Mother Culture's mould.\footnote{I use the term \emph{Mother Culture} to mean
a culture's most influencing features (its attitudes, values, viewpoints, etc.)
that are usually not consciously recognized as being culturally-specific by the
members of that culture \citep{quinn1992}.} 

As should have become apparent, we cannot escape the cages of our biological
and cultural programming. But, we can recognize that, unlike our biological
heritage, our cultural programming is not an immutable fact of life.
Potentially, our culture can be reprogrammed to the advantage of shared
(biological) values, and it probably will, as soon as more artists start to
catch up to these facts.

\paragraph{The suffering artist}

It is the job of the artist to invent new representations of reality. The artist's
fluency in expressing his or her thoughts and feelings, however, is no guaranty
that these will be particularly accurate (or even original). Still, scientists
can produce an endless stream of evidence on the nature of the forces that
shape our altruism. Yet, none of this will enter the mainstream consciousness
without the help of artists. Worse still: scientific literacy will not aid with
the integration of such knowledge in our personal life, unless it is translated
into mimetic and mythic thought structures first.

As we look down the cliff's face in this age of genetic engineering, will we be
able to use some of our ingenuity to re-engineer our culture? Mother Culture
tells us that we need to evolve---become better persons---to allow society to
further progress \citep{quinn1992}. But, really, shouldn't we be looking for a
new mother and a new place in the world? Can't we make our groups reflect what
we \textit{want} to be rather than let them dictate who we ought to be? I'm
convinced that we can, if the poets of today dare to wake up from this
nightmare to dream up new narratives in which there is a place for the
actualization of our human wants and needs. Such a culture, where the role of
altruism is not to deny our own fragile nature, is a culture that I could trust
to sustain our species and our ecosystems for a little bit longer. One day, I'd
like to have a dream in which I realize that the person I'm expected to be has
become the person I am.



\bibliographystyle{plainnat}
\bibliography{evo}


\end{document}
