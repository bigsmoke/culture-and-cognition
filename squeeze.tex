\documentclass{article}

\usepackage[utf8]{inputenc}
\usepackage[round]{natbib}
\usepackage{amsmath}
\usepackage{hyperref}

\title{Caught in the middle\\---\\between genes and memes}
\author{Rowan Rodrik van der Molen}

% The essay has to cover 3-4 book chapters/papers,
% besides the required reading for the course.
%
% Idea: use endnotes/footnote for concepts that Eelco may be less familiar with.
%
% Idea: Our genetic evolution has undermined our democratic ability.
%
% The 1st and 2nd draft are also handed in.
%
% \section{Introduction}
% Introduce topic
%
% \section{Discussion}
% The main body of the text
% Main question: Should we include group selection in human pop. genetics?
% Use (only) 1-2 examples where individual-level selection doesn't apply.
%
% \section{Conclusion}
% should relate the discussion to the introduction

\begin{document}

\maketitle

\tableofcontents

\section{Introduction}
\label{sec:intro}

This essay explores the implications of being a cultural creature, the central
question being what it means to be a carrier of genes that have evolved to support
the existence and evolution of what \citet{dawkins1976} calls memes. This is
definitely a very personal question, for both you---the reader---and me---the
witer---and our individual answers will differ. Given that neither our genes
nor our memes will be identical, there's unlikely to be an answer that
is universally valid. What I can, however, do is to frame the question such
that it helps us to successfully navigate the tensions that exist between our
individual and group needs, our cultural demands and the genetic programming of
ourselves and other individals. The ultimate purpose of this exercise is to
gain a deeper understanding of why we do what we do. More hopefully (but also
more unlikely), the answer may include what we \emph{want} to do.

As unlikely as the latter outcome may be, it is the desire to know this---or
rather, my frustration with the metacognitive deceptions that have so far
gotten in the way of true self-knowledge---that led to my long-term fascination
with this subject. It all started with a dream.

I had a dream---a lucid dream, in which I was aware that I was dreaming. From
studying \citet{laberge1990}, I knew that, in this dream-world, I could do
anything I wanted to. Nothing happened, until something did happen---something
that I thought I wanted; it turned out I didn't. The ensuing confusion---that I
could no longer trust my personal narrative about myself and my
motivations---arose from a misappropriation of the ownership of my personal
history, which was never personal to begin with. It was always a cultural
product.

\section{Mimetic culture and group selection}
\label{sec:mimesis}

\citet{donald1991} provides an account of the gradual evolution of modern
cognition that achieves its gradualism by hypothesizing an intermediate stage
between episodic cognition, which we share with the other great apes, and
mythic cognition, which only humans posess. He calls this intermediate stage
\textit{mimetic} cognition.

According to \citet{donald1991}, the mimetic stage signalled the emergence of
group selection in human evolution as an selective force that runs in parallel
with natural selection and sexual selection. From there on, the individual was
constrained by selective pressures eminating from the group as well as from
the natural environment, or \emph{niche}, occupied by the group. Archaic
homonids were already social animals before the emergence of mimetic culture,
but only with the advent of mimetic culture did the interests of the group
start to override the interests of the individual. How this was not so before
deserves elaboration.

\subsection{The selfish gene}
\label{sec:genes}

For non-human animals, the argument that they've evolved to serve “the greater
good” is easily refuted, as has repeatedly been done by a number of authors
(e.g. \citealp{dawkins1976,}). \citet{dawkins1976} has done this most expertly by
means of reducing individuals to copying machines for their genes. Genes, then,
can encourage altruism whenever this serves them best. That is, those genes
that have evolved to---in concert---influence the behaviour of their host such
that the maximum number of these genes are present in the next generation will
be the genes that will preferentially survive. Altruism then makes perfect
sense if you consider that family members will typically share many genes, and
indeed altruistic tendencies are generally highly discriminative, with the
closest relatives being the most likely beneficiaries [TODO: insert citation]. This
type of altruism is said to be the product of \emph{kin-selection} and can
be seen in, for example, the banded mongoose \citep{gilchrist2004a,
gilchrist2004b, hodge2005}.

Another type of altruism that is fairly common in the animal kingdom is
reciprocal, or \emph{tit-for-tat}, altruism, proposed by \citet{trivers1971}
as a model for cooperation between unrelated individuals within groups
of primates, including humans. A requirement for this strategy, according to
Trivers, is that individuals posess a good enough memory for keeping track of who
owes them what and what they owe whom. Such detailed book keeping is necessary
to avoid cheating; cheaters will inevitably avoid to exploit weaknesses in
other individuals.

In human societies, the functioning of psychopaths offers a wonderful insight
the social consequences of cheating. When so-called \emph{successful
psychopaths} are not detected, they can victimize not only individuals, but
whole groups \citep{babiak1995, boddy2006, boddy2010, kirkman2005}. Obviously,
even for us modern-day social primates, being able to recognize such cheating
behaviour is essential for success.  Selfish in respect to the individual, who
may be forced into altruistic behaviour by such a gene.

\subsection{Mimesis and a new level of selection}

Before detailing the absence of group selection in non-humans, it is useful to
dwell on how group selection would affect population genetic models.

There are four forces 

Integrate group selection as drift at the level of individual selection.

Group selection is about more than the cultural alleles. Tool-use and technology greatly complicate things.

Mimetic culture \citet{donald1991}.

%Explain how we resemble ants in our interdependence, but that our behaviour is not genetically coded for.


% This is amplified by mythic culture

In population genetics, various levels of selection have been considered: gene-level, individual-level and group-level selection. Only individual-level selection is normally considered, although genes can always be considered selfish to some extent \citep{dawkins1976}. Mathematically, this only matters when genes are so selfish that they compete within an individual. Otherwise, it's more useful to think in terms of fitness not of a single gene, but in the relative fitness of an individual carrying that gene.

Group selection is hardly ever applied in practice, although it is recognized that there are some theoretical cases where it would apply.
% Theoretical cases where group selection applies

% natural selection,
% sexual selection,
% group selection,
% cultural selection

\section{Mythic culture and cultural selection}

\subsection{The selfish meme}
\label{sec:memes}

A meme, according to \citet{dawkins1976}, can be defined as any type of
cultural entity that an observer might consider a replicator.  Memes are
information that can be exchanged between brains and thus be transferred to the
next generation.

In mimetic culture, attitudes and skills could already be transferred between groups.
In mythic culture, the group is starting to play a less important role as memes can be transferred more efficiently.

\subsection{How myths select for believers}

The mythical-mimetic frameworks through which we navigate our artificial modern
lives do not much resemble.

\section{The altruistic artist}

It is the job of the artist to invent new representations of reality. The artist's
fluency in expressing his or her thoughts and feelings, however, is no guaranty
that these will be particularly accurate (or even original). Still, scientists can
produce an endless stream of evidence on the nature of the forces acting on our
altruism; none of this will enter the mainstream consicousness without the help of
artists. Worse still: scientific literacy will not aid with the integration of such
knowledge in our personal life, unless it is translated into mimetic and mythic
thought structures first.

But, why is it so important to be able to distnguish between the various types
of self-sacrifice that have been discussed in this essay? Because, in this century,
our survival as a species might come to depend on our ability to reclaim culture
as the type of communal property as which it began. Not just us, but all the
planet's ecological support systems have been converted by culture. This
\textit{enculturalization} process is still running on full steam, blind and
relentless as ever. There are two problems with this: (1) anthropocenic
cultural practises have pushed us fully into the sixth great extinction event
and (2) this culture drives people crazy.

The first problem is widely recognized and uncontroversial. The second problem,
however, is rarely recognized, and when it is, often attributed to the natural
state of things.

In this age of genetic engineering, can we re-engineer our culture?

Fourth-generation warfare.

Mother Culture tells us that we need to evolve. But, really, shouldn't we be
looking for a new mother and a new place in the world? Can't we make our groups
reflect what we \textit{want} to be rather than let them dictate who we ought to
be (and want to be).

\subsection{Killing yourself may be good for your genes}

Suicide as an adaptive strategy.
% http://blogs.scientificamerican.com/bering-in-mind/2010/10/11/is-killing-yourself-adaptive-that-depends-an-evolutionary-theory-about-suicide/

\subsection{Murdering your neighbours may be good for your memes}

It is not controversial that some of your genes (i.e. those shared not by \emph{them}) may be well-served by murdering the competitors of your group.

\subsection{Positive feedback loops between cultural and biological evolution}

\subsection{The paper tiger and other cultural stresses}

How culture has long driven (groups of) people to excell, but has squasched curiosity.

\subsubsection{}

This essay attempts to interrupt the dominant cultural narrative that the human
condition has been steadily improving ever since human cultures started
increasing in complexity---an accelerating process that was kicked in a higher
gear with every new social revolution, such as the agricultural revolution and
the industrial revolution \citep{botton2013}. By trying to explain why our
culture---that very thing which we derive meaning and identity from---doesn't
necessarily care much about us (and even less about our feelings), I'll
demonstrate why it was in our genes' best interest to make us care more about
our disinterested cultures than about our own (and each other's) personal
well-being.

As will soon become apparent, we cannot escape the cages of our biological and
cultural programming. But, we can recognize that, unlike our biological
heritage, our cultural programming is not an immutable fact of life.
Potentially, our culture can be reprogrammed to the advantage of shared
(biological) values, and it probably will, as soon as more artists start to
catch up to these facts.

\bibliographystyle{plainnat}
\bibliography{evo}


\end{document}
